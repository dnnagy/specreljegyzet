\documentclass[a4paper,12pt]{article}

\usepackage{amsmath}
\usepackage{amsfonts}
\usepackage[magyar]{babel}
\usepackage[T1]{fontenc}
\usepackage[utf8]{inputenc}
\usepackage{graphicx}
\usepackage{fixltx2e}
\usepackage{multirow}
\usepackage{hyperref}

\newcommand{\dd}[1]{\mathrm{d}#1}
\newcommand{\dk}{\partial^k}
\newcommand{\dkk}{\partial_k}
\newcommand{\dxk}{\dd x^k}
\newcommand{\dxkk}{\dd x_k}
\newcommand{\xk}{x^k}
\newcommand{\xkk}{x_k}
\newcommand{\xl}{x^{\ell}}
\newcommand{\xll}{x_{\ell}}
\newcommand{\A}{\mathrm{A}}
\newcommand{\gkl}{g_{k\ell}}
\newcommand*\Laplace{\mathop{}\!\mathbin\bigtriangleup}


%Itt a megujitott commandok vannak
%\renewcommand*\contentsname{Tartalom}

\title{\textbf{A speciális relativitáselmélet alapjai\\
\vspace{24pt}}}
\author{\textsl{dr.Dávid Gyula előadásai alapján Nagy Dániel}}

\date{2017}
\begin{document}
\maketitle
\pagebreak
\tableofcontents

\section{Bevezetés}
A jelen jegyzet célja, hogy megörökítse az előadásokon elhangzottakat.
\section{Jelölések}
Ha másképp nem mondjuk, a jegyzet a következő jelöléseket alkalmazza: \\
\begin{itemize}
\item $k, \ell, m, n, p, q, s, t$: indexek $0$-tól $3$-ig
\item $\alpha, \beta, \gamma, \mu, \nu$: indexek $1$-től $3$-ig
\item $\A,\mathrm{B}$: tetszőleges tenzorok
\item $\xk, \xkk$: helyvektor a téridőben
\item $v^k, v_k, w^k, w_k$: tetszőleges négyesvektorok
\item $\tau$: sajátidő
\item $u^k, u_k$: négyessebesség-vektor
\item $\Theta, \mathrm{T}$: energia-impulzus tenzor
\item $\Phi(x)$: skalármező, ahol $x=\xk$
\item $\A(x)$: tenzormező, ahol $x=\xk$
\item $\Lambda$: Lorentz-transzformáció mátrixa 3+1 dimenzióban
\item $\gkl$: metrikus tenzor
\item $\delta_{k\ell}$: Kronecker-delta
\item $\delta(x)$: Dirac-delta
\item $\varepsilon_{ijk}$: Levi-Civita szimbólum
\item $\mathbf{r}$: helyvektor 3 dimenzióban
\item $L$
\item $\mathcal{L}$
\end{itemize}

\pagebreak
\section{Matematikai bevezetés}
\subsection{Vektorterek}
Legyen $\mathbb{K}$ test (pl. $\mathbb{R}$ vagy $\mathbb{C}$), $\mathbb{V}$ egy halmaz. A $\mathbb{V}$ halmazt $\mathbb{K}$ test feletti
vektortérnek nevezzük, ha teljesülnek az alábbi tulajdonságok:
\begin{itemize}
\item $(\mathbb{V}, +)$ Abel-csoport, vagy kommutatív csoport:
\subitem $(\mathbf{v} + \w) \in \mathbb{V}, ~ \forall \mathbf{v}, \w \in \mathbb{V}$
\subitem $(\mathbf{v} + \w) = (\w + \mathbf{v}), ~ \forall \mathbf{v}, \w \in \mathbb{V}$
\subitem $\mathbf{u} + (\mathbf{v} + \w) =(\mathbf{u} + \mathbf{v}) + \w, ~ \forall \mathbf{u},\mathbf{v}, \w \in \mathbb{V}$
\subitem $\exists \mathbf{0} \in \mathbb{V} ~ \textrm{úgy, hogy} ~ \mathbf{0} + \mathbf{v} = \mathbf{v} + \mathbf{0} = \mathbf{v} ~ \forall \mathbf{v}\in\mathbb{V}$
\subitem $ \forall \mathbf{v}\in\mathbb{V}\textrm{-re} ~ \exists (-\mathbf{v}) ~ \textrm{úgy, hogy} ~  \mathbf{v}+(-\mathbf{v})=\mathbf{0}$

\item Értelmezett a skalárral való szorzás a következő szabályokkal:
\subitem $\lambda \mathbf{v} \in \mathbb V, ~ \forall \lambda \in \mathbb{K}, ~ \forall \mathbf{v} \in \mathbb{V}$
\subitem $(\mu + \lambda) \mathbf{v} = \mu \mathbf{v} + \lambda\mathbf{v} , ~ \forall \mu, \lambda \in \mathbb{K}, \mathbf{v} \in \mathbb{V}$
\subitem $\lambda (\mathbf{v} + \w) = \lambda \mathbf{v} + \lambda \w, ~ \forall \lambda \in \mathbb{K}, ~ \forall \mathbf{v}, \w \in \mathbb{V}$
\subitem $\mu (\lambda \mathbf{v}) = (\mu \lambda)\mathbf{v} , ~ \forall \mu, \lambda \in \mathbb{K},~ \forall \mathbf{v} \in \mathbb{V}$

\end{itemize}

\subsection{Vektortér duális tere}
\subsection{Minkowski-tér}


"Néma indexnek Einstein se érti a szavát"\\
Néma index konvenció:\\
\[ \xkk\xk = \xk\xkk \equiv \sum\limits_{k=0}^3 \xkk \xk \]
\[ x_\alpha x^\alpha = x^\alpha x_\alpha \equiv \sum\limits_{\alpha=1}^3 x^\alpha x_\alpha \]
Egy szinten levő azonos indexekre nem összegzünk automatikusan, ha összegezni kell egy szinten levő indexre, akkor kiírjuk a
szummát.
\[ \xk = \begin{pmatrix} ct \\ x^\alpha \end{pmatrix} = \begin{pmatrix} ct \\ x \\ y \\ z \end{pmatrix} \]
\[ \xkk = \begin{pmatrix} ct \\ x_\alpha \end{pmatrix} = \begin{pmatrix} ct \\ -x \\ -y \\ -z \end{pmatrix} \]
\[ \xkk = \gkl \xl \]
\[ \xk = g^{k\ell}\xll \]
\[ \gkl = \begin{pmatrix} 1 & ~ & ~ & ~ \\ ~ & -1 & ~ & ~ \\ ~ & ~ & -1 & ~ \\ ~ & ~ & ~ & -1 \end{pmatrix} \]
\[ \gkl g^{\ell m} = \delta_k^{~m} \]
\[ x^{k^{'}} = \Lambda^k_{~\ell} \xl \]

\[\partial^\alpha \hat{F}_{\alpha\beta}\]
\appendix
\section{Matematikai fogalmak}
\end{document}
