\documentclass[a4paper,12pt]{article}

\usepackage{amsmath}
\usepackage[magyar]{babel}
\usepackage[T1]{fontenc}
\usepackage[utf8]{inputenc}
\usepackage{graphicx}
\usepackage{fixltx2e}
\usepackage{multirow}
\usepackage{hyperref}

\newcommand{\dd}[1]{\mathrm{d}#1}
\newcommand{\dk}{\partial^k}
\newcommand{\dkk}{\partial_k}
\newcommand{\dxk}{\dd x^k}
\newcommand{\dxkk}{\dd x_k}
\newcommand{\xk}{x^k}
\newcommand{\xkk}{x_k}
\newcommand{\xl}{x^{\ell}}
\newcommand{\xll}{x_{\ell}}
\newcommand{\A}{\mathrm{A}}
\newcommand{\gkl}{g_{k\ell}}
\newcommand*\Laplace{\mathop{}\!\mathbin\bigtriangleup}

\newcommand{\vk}{v^k}
\newcommand{\vkk}{v_k}
\newcommand{\vl}{v^\ell}
\newcommand{\vll}{v_\ell}

\newcommand{\w}{\mathbf{w}}
\newcommand{\wk}{w^k}
\newcommand{\wkk}{w_k}
\newcommand{\wl}{w^\ell}
\newcommand{\wll}{w_\ell}


%Itt a megujitott commandok vannak
%\renewcommand*\contentsname{Tartalom}

\title{\textbf{A speciális relativitáselmélet alapjai\\
\vspace{24pt}}}
\author{\textsl{dr.Dávid Gyula előadásai alapján Nagy Dániel}}

\date{2017}
\begin{document}
\maketitle
\pagebreak
\tableofcontents

\section{Bevezetés}
A jelen jegyzet célja, hogy megörökítse az előadásokon elhangzottakat.
\section{Jelölés}
Ha másképp nem mondjuk, a jegyzet a következő jelöléseket alkalmazza: \\
\begin{itemize}
\item $k, \ell, m, n, p, q, s, t$: indexek $0$-tól $3$-ig
\item $\alpha, \beta, \gamma, \mu, \nu$: indexek $1$-től $3$-ig
\item $\A,\mathrm{B}$: tetszőleges tenzorok
\item $\xk, \xkk$: helyvektor a téridőben
\item $u^k, v^k, u_k, v_k$: tetszőleges négyesvektorok
\item $\tau$: sajátidő
\item $\Theta, \mathrm{T}$: energia-impulzus tenzor
\item $\Phi(x)$: skalármező, ahol $x=\xk$
\item $\A(x)$: tenzormező, ahol $x=\xk$
\item $\Lambda$: Lorentz-transzformáció mátrixa 3+1 dimenzióban
\item $\gkl$: metrikus tenzor
\item $\delta_{k\ell}$: Kronecker-delta
\item $\delta(x)$: Dirac-delta
\item $\varepsilon_{ijk}$: Levi-Civita szimbólum
\end{itemize}

\pagebreak
\section{Matematikai bevezetés}
$$\xk = \begin{pmatrix} ct \\ x^\alpha \end{pmatrix} = \begin{pmatrix} ct \\ x \\ y \\ z \end{pmatrix}$$
$$\xkk = \gkl \xl$$
$$\xk = g^{k\ell}\xll$$
\appendix
\section{Matematikai fogalmak}
\end{document}
