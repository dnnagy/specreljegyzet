\subsection{Vektorterek}
Legyen $\mathbb{K}$ test (pl. $\mathbb{R}$ vagy $\mathbb{C}$), $\mathbb{V}$ egy halmaz. A $\mathbb{V}$ halmazt $\mathbb{K}$ test feletti
vektortérnek nevezzük, ha teljesülnek az alábbi tulajdonságok:
\begin{itemize}
\item $(\mathbb{V}, +)$ Abel-csoport, vagy kommutatív csoport:
\subitem $(\mathbf{v} + \w) \in \mathbb{V}, ~ \forall \mathbf{v}, \w \in \mathbb{V}$
\subitem $(\mathbf{v} + \w) = (\w + \mathbf{v}), ~ \forall \mathbf{v}, \w \in \mathbb{V}$
\subitem $\mathbf{u} + (\mathbf{v} + \w) =(\mathbf{u} + \mathbf{v}) + \w, ~ \forall \mathbf{u},\mathbf{v}, \w \in \mathbb{V}$
\subitem $\exists \mathbf{0} \in \mathbb{V} ~ \textrm{úgy, hogy} ~ \mathbf{0} + \mathbf{v} = \mathbf{v} + \mathbf{0} = \mathbf{v} ~ \forall \mathbf{v}\in\mathbb{V}$
\subitem $ \forall \mathbf{v}\in\mathbb{V}\textrm{-re} ~ \exists (-\mathbf{v}) ~ \textrm{úgy, hogy} ~  \mathbf{v}+(-\mathbf{v})=\mathbf{0}$

\item Értelmezett a skalárral való szorzás a következő szabályokkal:
\subitem $\lambda \mathbf{v} \in \mathbb V, ~ \forall \lambda \in \mathbb{K}, ~ \forall \mathbf{v} \in \mathbb{V}$
\subitem $(\mu + \lambda) \mathbf{v} = \mu \mathbf{v} + \lambda\mathbf{v} , ~ \forall \mu, \lambda \in \mathbb{K}, \mathbf{v} \in \mathbb{V}$
\subitem $\lambda (\mathbf{v} + \w) = \lambda \mathbf{v} + \lambda \w, ~ \forall \lambda \in \mathbb{K}, ~ \forall \mathbf{v}, \w \in \mathbb{V}$
\subitem $\mu (\lambda \mathbf{v}) = (\mu \lambda)\mathbf{v} , ~ \forall \mu, \lambda \in \mathbb{K},~ \forall \mathbf{v} \in \mathbb{V}$

\end{itemize}

\subsection{Vektortér duális tere}
Vegyünk most két vektorteret, jelöljük őket $\mathbb{V}$-vel illetve $\mathbb{W}$-vel. Az egyszerűség kedvéért
legyen mindkét vektortér az $\mathbb{R}$ valós számtest felett értelmezve. Ekkor a $\mathbf{v} \in \mathbb{V}$ -hez rendelhetünk
egy $\mathbf{w} \in \mathbb{W}$ elemet a következő összefüggéssel:
\[ \mathbf{w} = \hat{A}\mathbf{v} \]
ahol $\hat{A}$ egy $\mathbb{V} \rightarrow \mathbb{W}$ típusú leképezés.
Ha megköveteljük, hogy az $\hat{A}$ leképezés legyen lineáris, akkor érvényes, hogy
\[ \hat{A}(\mathbf{v_1} + \mathbf{v_2}) = \hat{A}\mathbf{v_1} + \hat{A}\mathbf{v_2} \]
és
\[ \hat{A}(\alpha\mathbf{v}) = \alpha(\hat{A}\mathbf{v}) \]
Tekintsünk most két bázist: $\left \{ \mathbf{e}_{(k)} \right \} _{k=1}^n \in \mathbb{V}$ és $\left \{ \mathbf{f}_{(\ell)} \right \} _{\ell=1}^m \in \mathbb{W}$.
Ekkor írhatjuk, hogy
\[ \mathbf{v} = v^k\mathbf{e}_{(k)} \equiv \sum\limits_{k=1}^n v^k\mathbf{e}_{(k)}  \]
\\ Vagyis a $\mathbf{v}$ vektor az $\left \{ \mathbf{e}_{(k)} \right \} _{k=1}^n \in \mathbb{V}$ bázison kifejthető a
$v^k$ koordináták segítségével. Hasonlóan írhatjuk, hogy
\[ \mathbf{w} = w^\ell \mathbf{f}_{(\ell)} \equiv \sum\limits_{\ell=1}^m w^\ell \mathbf{f}_{(\ell)}  \]
ahol használjuk a felsőindex-konvenciót és az Einstein féle összegzési konvenciót.
Hasson most egy $\hat{A}$ lineáris operátor a $\mathbf{v}$ vektorra.
Ekkor
\[ \hat{A}\mathbf{v} = \hat{A} \left( v^k\mathbf{e}_{(k)} \right) = v^k\left( \hat{A}\mathbf{e}_{(k)} \right) \]
Tehát ha ismerjük a bázisvektorok képét, akkor bármely $\mathbf{v}$ vektor képét ismerjük, hiszen $\mathbf{v}$ kifejthető
a bázison.
\\ De mivel $\hat{A}\mathbf{e}_{(k)} \in \mathbb{W}$, ezért kifejthető az $\left \{ \mathbf{f}_{(\ell)} \right \} _{\ell=1}^m \in \mathbb{W}$ bázison:
\[ \hat{A}\mathbf{e}_{(k)} = \sum\limits_{\ell} \A_{~k}^{\ell} \mathbf{f}_{(\ell)} \textrm{ , ahol } \A_{~k}^{\ell} \in \mathbb{R} \]

De tudjuk, hogy $\hat{A}\mathbf{v} = \mathbf{w}$, ezért a fentiek szerint
\[ \mathbf{w} = \hat{A}\mathbf{v} = v^k\left( \hat{A}\mathbf{e}_{(k)} \right) = v^k \A_{~k}^{\ell} \mathbf{f}_{(\ell)} \]
ami nem más mint
\[ \sum\limits_{\ell} \left( \sum\limits_{k} \A_{~k}^{\ell} v^k \right) \mathbf{f}_{(\ell)}\]
\subsection{Minkowski-tér}
